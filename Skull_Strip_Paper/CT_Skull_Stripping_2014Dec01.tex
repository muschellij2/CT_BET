\documentclass{elsarticle}\usepackage[]{graphicx}\usepackage[]{color}
%% maxwidth is the original width if it is less than linewidth
%% otherwise use linewidth (to make sure the graphics do not exceed the margin)
\makeatletter
\def\maxwidth{ %
  \ifdim\Gin@nat@width>\linewidth
    \linewidth
  \else
    \Gin@nat@width
  \fi
}
\makeatother

\definecolor{fgcolor}{rgb}{0.345, 0.345, 0.345}
\newcommand{\hlnum}[1]{\textcolor[rgb]{0.686,0.059,0.569}{#1}}%
\newcommand{\hlstr}[1]{\textcolor[rgb]{0.192,0.494,0.8}{#1}}%
\newcommand{\hlcom}[1]{\textcolor[rgb]{0.678,0.584,0.686}{\textit{#1}}}%
\newcommand{\hlopt}[1]{\textcolor[rgb]{0,0,0}{#1}}%
\newcommand{\hlstd}[1]{\textcolor[rgb]{0.345,0.345,0.345}{#1}}%
\newcommand{\hlkwa}[1]{\textcolor[rgb]{0.161,0.373,0.58}{\textbf{#1}}}%
\newcommand{\hlkwb}[1]{\textcolor[rgb]{0.69,0.353,0.396}{#1}}%
\newcommand{\hlkwc}[1]{\textcolor[rgb]{0.333,0.667,0.333}{#1}}%
\newcommand{\hlkwd}[1]{\textcolor[rgb]{0.737,0.353,0.396}{\textbf{#1}}}%

\usepackage{framed}
\makeatletter
\newenvironment{kframe}{%
 \def\at@end@of@kframe{}%
 \ifinner\ifhmode%
  \def\at@end@of@kframe{\end{minipage}}%
  \begin{minipage}{\columnwidth}%
 \fi\fi%
 \def\FrameCommand##1{\hskip\@totalleftmargin \hskip-\fboxsep
 \colorbox{shadecolor}{##1}\hskip-\fboxsep
     % There is no \\@totalrightmargin, so:
     \hskip-\linewidth \hskip-\@totalleftmargin \hskip\columnwidth}%
 \MakeFramed {\advance\hsize-\width
   \@totalleftmargin\z@ \linewidth\hsize
   \@setminipage}}%
 {\par\unskip\endMakeFramed%
 \at@end@of@kframe}
\makeatother

\definecolor{shadecolor}{rgb}{.97, .97, .97}
\definecolor{messagecolor}{rgb}{0, 0, 0}
\definecolor{warningcolor}{rgb}{1, 0, 1}
\definecolor{errorcolor}{rgb}{1, 0, 0}
\newenvironment{knitrout}{}{} % an empty environment to be redefined in TeX

\usepackage{alltt}
\usepackage[top=1in, left=1in, right=1in, bottom=1in]{geometry}

\usepackage{float, amsmath}
\usepackage{tikz}
\usetikzlibrary{shapes,arrows}
\usetikzlibrary{positioning}
\usepackage{float, amsmath}

\usepackage[hyphens]{url}
\usepackage{enumerate}
% \usepackage{natbib}


%\usepackage[
%  natbib = true,
%    backend=bibtex,
%    isbn=false,
%    url=false,
%    doi=false,
%    eprint=false,
%    style=numeric,
%    sorting=nyt,
%    sortcites = true
%]{biblatex}
%\bibliography{CT_Pipeline_l}
%\bibliography{CT_Skull_Stripping}

\usepackage{hyperref}

\makeatletter
\providecommand{\doi}[1]{%
  \begingroup
    \let\bibinfo\@secondoftwo
    \urlstyle{rm}%
    \href{http://dx.doi.org/#1}{%
      doi:\discretionary{}{}{}%
      \nolinkurl{#1}%
    }%
  \endgroup
}
\makeatother


\usepackage{subfig}

\journal{NeuroImage}













%\usepackage[all]{hypcap}
\IfFileExists{upquote.sty}{\usepackage{upquote}}{}
\begin{document}
\renewcommand{\thesubfigure}{\Alph{subfigure}}

\begin{frontmatter}

\date{}

\title{Validated Automatic Brain Extraction of Head CT Images}
%\title{Validated Automatic Brain Extraction of Head CT Images using Established, Open-Source, Neuroimaging Software}

% \author[jhsph]{John~Muschelli\corref{cor1}}
% \ead{jmusche1@jhu.edu}
% 
% \author[jhmi]{Natalie~Ullman}
% \ead{nullman1@jhmi.edu}
% 
% \author[jhmi]{Daniel~F.~Hanley}
% \ead{dhanley@jhmi.edu}
% 
% \author[jhsph]{Ciprian~M.~Crainiceanu}
% \ead{ccrainic@jhsph.edu}
% 
% 
% \cortext[cor1]{Principal Corresponding Author}
% \address[jhsph]{Department of Biostatistics, Bloomberg School of Public Health, Johns Hopkins University, Baltimore, MD, USA}
% \address[jhmi]{Department of Neurology, Division of Brain Injury Outcomes,  Johns Hopkins Medical Institutions, Baltimore, MD, USA}


\author[jhsph]{John~Muschelli\corref{cor1}}
\ead{jmusche1@jhu.edu}

\author[jhmi]{Natalie~L.~Ullman}
\ead{nullman1@jhmi.edu}

\author[ucla]{Paul~Vespa}
\ead{PVespa@mednet.ucla.edu}


\author[jhmi]{Daniel~F.~Hanley}
\ead{dhanley@jhmi.edu}

\author[jhsph]{Ciprian~M.~Crainiceanu}
\ead{ccrainic@jhsph.edu}


\cortext[cor1]{Principal Corresponding Author}
\address[jhsph]{Department of Biostatistics, Bloomberg School of Public Health, Johns Hopkins University, Baltimore, MD, USA}
\address[jhmi]{Department of Neurology, Division of Brain Injury Outcomes,  Johns Hopkins Medical Institutions, Baltimore, MD, USA}
\address[ucla]{Department of Neurosurgery, David Geffen School of Medicine at UCLA, Los Angeles, CA, USA}






\begin{abstract}
\section*{Background}
Computed Tomography (CT) imaging of the brain is commonly used in diagnostic settings.  Although CT scans are primarily used in clinical practice, they are increasingly used in research.  A fundamental processing step in brain imaging research is brain extraction -- the process of separating the brain tissue from all other tissues. Methods for brain extraction in head CT images have been informally proposed, but never formally validated.


\section*{Aim}

To systematically analyze and validate the performance of FSL's brain extraction tool (BET) on head CT images of patients with intracranial hemorrhage.  This was done by comparing the manual gold standard with the results of several versions of automatic brain extraction and by estimating the reliability of automated segmentation of longitudinal scans.  The effects of the choice of BET parameters and data smoothing is studied and reported.

\section*{Methods}

All images were thresholded using a $0-100$ Hounsfield units (HU) range. In one variant of the pipeline, data were smoothed using a 3-dimensional Gaussian kernel ($\sigma=1$mm$^3$) and re-thresholded to $0-100$ HU; in the other, data were not smoothed.  BET was applied using 1 of 3 fractional intensity (FI) thresholds: $0.01$, $0.1$, or $0.35$ and any holes in the brain mask were filled.

For validation against a manual segmentation, 36 images from 36 different patients with intracranial hemorrhage were selected from 19 different MISTIE (Minimally Invasive Surgery plus recombinant-tissue plasminogen activator for Intracerebral Evacuation) stroke trial centers. Intracranial masks of the brain were manually created by one expert CT reader.  The resulting brain tissue masks were quantitatively compared to the manual segmentations using sensitivity, specificity, accuracy, and the Dice Similarity Index (DSI).  Brain extraction performance across smoothing and FI thresholds was compared using the Wilcoxon signed-rank test.  

To estimate the performance in a large number of scans,  brain masks were generated from the $6$ BET pipelines for 1095 longitudinal scans from 129 patients.  The intracranial volume (ICV) of each scan was estimated by multiplying the number of voxels in the brain mask by the dimensions of each voxel for that scan.  Failure rates were estimated from visual inspection and an intraclass correlation (ICC) of ICV was estimated using a one-way ANOVA.

\section*{Results}

Smoothing images improves brain extraction results using BET for all measures except specificity (all $p < 0.01$, uncorrected), irrespective of the FI threshold.  Using an FI of $0.01$ or $0.1$ performed better than $0.35$.  Thus, all reported results refer only to smoothed data using an FI of $0.01$ or $0.1$.  Using an FI of $0.01$ had a higher median sensitivity ($0.9901$) than an FI of $0.1$ ($0.9884$, median difference: $0.0014$, $p< 0.001$), accuracy ($0.9971$ vs. $0.9971$; median difference: $0.0001$, $p< 0.001$), and DSI ($0.9895$ vs. $0.9894$; median difference: $0.0004$, $p< 0.001$) and lower specificity ($0.9981$ vs. $0.9982$; median difference: $-0.0001$, $p< 0.001$).  These measures are all very high indicating that a range of FI values may produce visually indistinguishable brain extractions.  Using smoothed data and an FI of $0.01$ had a low failure rate ($5.2$\%) and the ICC estimate was high (0.929, 95\% CI: 0.91, 0.945) for successfully extracted brains.

\section*{Conclusion}

BET performs well at brain extraction on thresholded, $1$mm$^3$ smoothed CT images with an FI of $0.01$ or $0.1$.  Smoothing before applying BET is an important step not previously discussed in the literature.  Analysis code is provided.

\end{abstract}

\begin{keyword}
CT \sep skull stripping \sep brain extraction \sep validation
\end{keyword}

\end{frontmatter}



\section{Introduction}

X-ray computed tomography (CT) scanning of the brain is widely available and is a commonly used diagnostic tool in clinical settings \citep{sahni_management_2007, chalela2007magnetic, schellinger1999standardized}. Though analysis of CT images is typically done by qualitative visual inspection, detailed quantification of information using neuroimaging tools is of interest.  The reason for this interest is that qualitative inspection of CT scans provides limited quantifiable information that can be used in research. A fundamental processing step for producing quantifiable and  reproducible information about the brain is to extract the brain from the CT image. This process is called brain extraction or skull stripping.  This step is necessary because CT images often contain non-brain human tissues (e.g. skull, eyes, skin) or non-human elements (e.g. pillow, medical devices) that are not pertinent to brain research.  We propose a validated automated solution to brain extraction in head CT scans using established neuroimaging software.

In magnetic resonance imaging (MRI), brain extraction has been extensively studied and investigated (see \citet{wang2014knowledge} for an overview of methods).  While an extensive literature accompanied by software exist for brain MRI scans, the same is not true for brain CT scans.  \citet{smith_fast_2002} introduced and validated the Brain Extraction Tool (BET), a function of the FSL \citep{jenkinson_fsl_2012} neuroimaging software (v5.0.4), to automatically extract the brain from MRI scans.  Here we propose to adapt BET and validate its brain extraction performance for CT scans.  Such BET adaptations for this purpose have been presented before in \citet{solomon_user-friendly_2007} and have been discussed in more detail in \citet{rorden_age-specific_2012}.  Neither presented a formal validation against a set of manually segmented brain images nor estimated the performance in a large number of CT scans, which are the goals of our study. 



\section{Methods}
\subsection{ Participants and CT data}
We used CT images from patients enrolled in the MISTIE (Minimally Invasive Surgery plus recombinant-tissue plasminogen activator for Intracerebral Evacuation) and ICES (Intraoperative CT-Guided Endoscopic Surgery) stroke trials \citep{morgan_preliminary_2008}.  These patients had an intracranial hemorrhage at the time of scanning; for inclusion criteria, see \citet{mould_minimally_2013}.   CT data were collected as part of the Johns Hopkins Medicine IRB-approved MISTIE research studies with written consent from participants.  


\subsection{Imaging Data}
\subsubsection{Validation of Automated Head Segmentation}
For the validation of automated segmentation against the gold standard, we analyzed 36 scans, corresponding to 36 unique patients.  The study protocol was executed with minor, but important, differences across the 19 sites.  Scans were acquired using Siemens ($N=14$), GE ($N=11$), Philips ($N=10$), and Toshiba ($N=1$) scanners. Gantry tilt was observed in 21 scans.  Slice thickness of the image varied within the scan for 7 scans. For example, a scan may have 10 millimeter (mm) slices at the top and bottom of the brain and 5mm slices in the middle of the brain.  Therefore, the scans analyzed had different voxel (volume element) dimensions and image resolution prior to registration to the template.  These conditions represent how scans are presented for evaluation in many diagnostic cases.  

%n_crani_problem 
%n_gantry_problem 


\subsection{Manual and Automated Brain Extraction}
Brain tissue was manually segmented as a binary mask from DICOM (Digital Imaging and Communications in Medicine) images using the OsiriX imaging software (OsiriX v.4.1, Pixmeo; Geneva, Switzerland) by one expert reader. 
CT brain images and the binary brain tissue mask obtained using manual segmentation were exported from OsiriX to DICOM format.  

\subsection{Image Processing}
The image processing pipeline is provided in Figure~\ref{fig:framework}.
Images with gantry tilt were corrected using a customized MATLAB (The Mathworks, Natick, Massachusetts, USA) user-written script (\url{http://bit.ly/1ltIM8c}).  Although gantry tilt correction is not inherently necessary for brain extraction, it is required for rigid co-registration of scans within a patient, which is a common processing step in longitudinal analysis of images post brain extraction. 

Images were converted to the Neuroimaging Informatics Technology Initiative (NIfTI) data format using \texttt{dcm2nii} (2009 version, provided with MRIcro \citep{rorden_stereotaxic_2000}).  Images were constrained to values between $-1024$ and $3071$ HU to remove potential image rescaling errors and artifacts.  No interpolation was done for images with a variable slice thickness. Thickness was determined from the first converted slice and was assumed homogeneous throughout the image.  This assumption should have no affect on the estimates of performance described below, as they are calculated at a voxel level, except for volume estimation because volume estimates can be off by an order of magnitude if variable slice thickness is not properly accounted for.


\tikzstyle{bblock} = [rectangle, draw, text width=8em, text centered, minimum height=2em, rounded corners]
\tikzstyle{line} = [draw, text centered , -latex']
\tikzstyle{line node} = [draw, fill=white, font=\tiny ]
\tikzstyle{block} = [rectangle, draw, text width=5em, text centered, minimum height=4em, rounded corners]    

\begin{figure}
\centering
\begin{tikzpicture}[node distance = 2cm, every node/.style={rectangle,fill=white}, scale=0.75, transform shape]
% Place nodes
\node [bblock] (raw) {DICOM images};
\node [bblock, below = 2.5cm of raw] (dcmnii) {NIfTI image};
\node [bblock, below of=dcmnii] (thresh) {Threshold to 0-100 HU };
\node [bblock, above right=1cm and 1.25cm of dcmnii] (gantry) {Gantry tilt correction};
\node [bblock, below of=thresh, left of=thresh] (nosmooth) {No Smooth};
\node [bblock, below of=thresh, right of=thresh] (smooth) {Smooth ($1$mm$^3$)};

\node [bblock, below of=nosmooth, right of=nosmooth] (BET) {BET};

\node [block, below of=BET, node distance = 3cm] (SS_1) {FI=0.1};
\node [block, left = 1.75em of SS_1] (SS_01) {FI=0.01};
\node [block, right = 1.75em of SS_1] (SS_35) {FI=0.35};

\node [block, below of=SS_1] (Mask) {Threshold image $> 0$ (Mask)};

\node [block, below of=Mask, node distance = 2cm] (Fillin) {Fill mask holes};

\node [bblock, below of=Fillin, node distance = 2cm] (Measures) {Performance Measures};


% Draw edges
\path [line] (raw) -- node {dcm2nii} (dcmnii);
\path [line] (raw) -- (gantry);
\path [line] (gantry) -- node {dcm2nii} (dcmnii);
\path [line] (dcmnii) -- (thresh);
\path [line] (thresh) -- (nosmooth);
\path [line] (thresh) -- (smooth);
\path [line] (smooth) -- (BET);
\path [line] (nosmooth) -- node[above right= -0.15cm and -0.6cm of BET] {Threshold $0-100$ HU} (BET);
\path [line] (BET) -- (SS_01);
\path [line] (BET) -- (SS_35);
\path [line] (BET) -- node {Different fractional intensity (FI) Value} (SS_1);
\path [line] (SS_1) -- (Mask);
\path [line] (SS_01) -- (Mask);
\path [line] (SS_35) -- (Mask);
\path [line] (Mask) -- (Fillin);
\path [line] (Fillin) -- (Measures);
\end{tikzpicture}
\caption{{\bf Processing Pipeline}.  Images in DICOM (Digital Imaging and Communications in Medicine) format were gantry tilt corrected if necessary and converted to NIfTI (Neuroimaging Informatics Technology Initiative) format using \texttt{dcm2nii}.  After NIfTI conversion, the data is thresholded to tissue ranges of $0-100$ Hounsfield units (HU).  In one variant of the pipeline, the data was smoothed using a 3-dimensional Gaussian kernel ($\sigma=1$mm$^3$) and re-thresholded to $0-100$ HU; in the other, the data was not smoothed.  BET was applied to the image using 3 different fractional intensity (FI) values: $0.01$, $0.1$, and $0.35$.  The resultant image was masked to values greater than $0$ HU and FSL was used to fill in any holes.  These filled masks were used in comparison to the manually segmented image.  }
\label{fig:framework}
\end{figure}

Each image was thresholded using the brain tissue range ($0-100$ HU).  In one variant of the pipeline, data were smoothed using a 3-dimensional (3D) Gaussian kernel ($\sigma=1$mm$^3$) and re-thresholded to $0-100$ HU; in the other, data were not smoothed.  BET was then applied, varying the fractional intensity (FI) parameter to determine its influence on performance: we used values of $0.35$ (as used in \citet{rorden_age-specific_2012}), $0.1$, and $0.01$.  The FI parameter varies between $0$ and $1$ and determines the location of the edge of the segmented brain image; smaller values correspond to larger brain masks. After BET was applied, we created a brain mask taking values $> 0$ HU and filled the holes in the mask (using \verb|fslmaths -fillh|).  


\subsection{Measuring and Testing Brain Extraction Performance}
We compared the masks obtained using the various choices of parameters to the manually segmented images.  Four common measurements of performance were calculated for each image: sensitivity, specificity, accuracy, and the Dice Similarity Index (DSI) \citep{dice_measures_1945}.  For each measure, higher values indicate better agreement with the manual segmentation.  See Inline Supplementary Methods 1 for the calculation of each measure.

[Insert Supplementary Methods 1 here]

We calculated the paired difference of each measure using different pipelines (e.g. $0.01$ vs. $0.1$, smoothed data).  We tested whether these differences were statistically different from zero using the Wilcoxon signed-rank test.



\subsection{Failure Rate and Intraclass Correlation Estimate}
Although comparison of automated methods to a manual gold standard is ideal, manual segmentation requires a significant amount of time.  Therefore, for a large number of scans, this procedure is impractical.  As multiple CT scans are obtained from patients in the MISTIE trial, we can estimate the reliability of our proposed brain extraction pipelines without manual segmentation by comparing intracranial volumes of the same patient on subsequent scans.  Moreover, we can estimate failure rate of each pipeline.  

For these tasks, we collected 1160 scans.  Of these scans, we excluded 27 scans due to craniotomy and 38 due to the gantry tilt correction forcing areas of the brain outside the field of view.   We executed the previous brain extraction pipelines on the remaining 1095 scans.  Of these scans, we visually assessed the quality of brain extraction: any scan excluding a significant portion of the brain or having holes due to mask self-intersection were classified as a failure.  These scans represent 129 patients from 26 sites, with a mean (SD) of 8.5 (2.8) scans per patient.  Scans were acquired using Siemens ($N=492$), GE ($N=298$), Philips ($N=207$), Toshiba ($N=66$), Neurologica ($N=30$), and Picker ($N=2$) scanners.  We estimated the failure rate for each processing pipeline and used a Fisher's exact test to test whether failure rates differed across scanners.

From each scan, we calculated the intracranial volume (ICV) by multiplying the number of voxels in the resulting brain mask by the dimensions of each voxel.  In scans with variable slice thickness, we multiplied the in-plane resolution of each voxel by the slice thickness given from the ImagePositionPatient field in the DICOM header to correctly account for the different voxel dimensions.  

Using only the scans with successful brain extraction, we estimated the intraclass correlation (ICC) and its confidence interval (CI) using a one-way ANOVA, where patients are the groups, for unbalanced repeated measures \citep{searle_linear_2012, thomas_interval_1978, donner_use_1979, lessells_unrepeatable_1987} using the ICC package \citep{wolak_guidelines_2012} in R (\url{http://cran.r-project.org/}).  

%and the Jaccard index is defined as:
%$$
%\frac{ \sum_{i=1}^{V} \left( I_{ia} \times I_{im}\right) }{\sum_{i=1}^{V} I_{ia}  + \sum_{i=1}^{V} I_{im} - \sum_{i=1}^{V} \left(I_{ia} \times I_{im} \right)}
%$$



\section{Results}
\subsection{Manual and Automated Brain Extraction}
Figure~\ref{fig:metrics}\protect\subref*{unsmoothed} illustrates the performance of each variation of the BET pipeline in Figure~\ref{fig:framework}.  The pipelines using smoothing (top panel) perform better than the unsmoothed pipelines (bottom panel) on all measures except specificity (all $p < 0.01$, uncorrected for multiplicity).  BET also performed poorly on some scans without smoothing.  

Figure~\ref{fig:metrics}\protect\subref*{smoothed} displays the performance for brain extraction for the pipelines using smoothed images.   Because the performance for all metrics was high when using smoothed images, it was necessary to change the y-axis from $[0,1]$ to $[0.95,1]$. 
Using an FI of $0.01$ or $0.1$ performed better than $0.35$; thus, we will focus and compare results for these values of FI only for the case when BET was applied to smoothed images.  Using an FI of $0.01$ had a higher median sensitivity ($0.9901$) than an FI of $0.1$ ($0.9884$, median difference: $0.0014$, $p< 0.001$), accuracy ($0.9971$ vs. $0.9971$; median difference: $0.0001$, $p< 0.001$), and DSI ($0.9895$ vs. $0.9894$; median difference: $0.0004$, $p< 0.001$) and lower specificity ($0.9981$ vs. $0.9982$; median difference: $-0.0001$, $p< 0.001$).  Overall, regardless of p-values, these measures are all very high, indicating that multiple choices of parameters work well for brain extraction after CT image processing.  Moreover, a Bonferroni correction for multiple comparisons yields the same conclusions. 

% Overall, using an FI of $0.35$ performs worse overall than $0.01$ or $0.1$ for all measures other than specificity.  
% 
% Without smoothing the images, BET performed poorly regardless of FI.  
% 












\begin{figure}
  \subfloat{
  \label{unsmoothed}
\includegraphics[width=.48\textwidth]{figure/CT_Skull_Stripping_Figure2.png}
}
\hfill
  \subfloat{
  \label{smoothed}
\includegraphics[width=.48\textwidth]{figure/CT_Skull_Stripping_Figure2b.png}
}
\caption{{\bf Performance Metric Distribution for Different Pipelines.} Panel~\protect\subref*{unsmoothed} displays the boxplots for performance measures when running the pipeline with a different fractional intensity (FI), using smoothed data (top) or unsmoothed data (bottom).  Panel~\protect\subref*{smoothed} presents the smoothed data only, rescaled to show discrimination between the different FI. Overall, FI of $0.01$ and $0.1$ perform better than $0.35$ in all categories other than specificity.  Using smoothed data improves performance in all performance metrics, markedly when an FI of $0.35$ is used.  Panel~\protect\subref*{smoothed} demonstrates that using an FI of $0.01$ on smoothed data has high performance on all measures.  }
\label{fig:metrics}
\end{figure}



\begin{figure}[htb]
\centering
  \subfloat{
  \label{ss:01_smooth}
	\includegraphics[width=.315\textwidth]{figure/{101-307_20061110_1638_CT_5_RM_Head_SS_0.01_Mask}.png} 
}
\hfill
  \subfloat{
  \label{ss:1_smooth}
	\includegraphics[width=.315\textwidth]{figure/{101-307_20061110_1638_CT_5_RM_Head_SS_0.1_Mask}.png} 
}
\hfill
  \subfloat{
  \label{ss:35_smooth}
	\includegraphics[width=.315\textwidth]{figure/{101-307_20061110_1638_CT_5_RM_Head_SS_0.35_Mask}.png} 
}
\newline
\hfill 
  \subfloat{
  \label{ss:01_nosmooth}
	\includegraphics[width=.315\textwidth]{figure/{101-307_20061110_1638_CT_5_RM_Head_SS_0.01_nopresmooth_Mask}.png} 
}
\hfill
  \subfloat{
  \label{ss:1_nosmooth}
	\includegraphics[width=.315\textwidth]{figure/{101-307_20061110_1638_CT_5_RM_Head_SS_0.1_nopresmooth_Mask}.png} 
}
\hfill
  \subfloat{
  \label{ss:35_nosmooth}
	\includegraphics[width=.315\textwidth]{figure/{101-307_20061110_1638_CT_5_RM_Head_SS_0.35_nopresmooth_Mask}.png} 
}
\caption{{\bf Example Case where Smoothing before BET is Required.} For one subject, the CT image is displayed with the brain extracted mask in red after running all pipelines.  Panels~\protect\subref*{ss:01_smooth}, \protect\subref*{ss:1_smooth}, and \protect\subref*{ss:35_smooth} represent applying BET using FI of $0.01$, $0.1$, and $0.35$, respectively, to smoothed data. Panels~\protect\subref*{ss:01_nosmooth}, \protect\subref*{ss:1_nosmooth}, and \protect\subref*{ss:35_nosmooth} correspond to applying BET using FI $0.01$, $0.1$, and $0.35$ on unsmoothed data.  Smoothing images improves brain extraction with BET.
}
\label{fig:ss_example}
\end{figure}



\subsection{Failure Rate and Intraclass Correlation Estimate}
Although Figure~\ref{fig:metrics} indicates that using FI of $0.01$ or $0.1$ provides adequate brain extraction results for most cases, they perform relatively well regardless whether or not the data are smoothed.  Figure~\ref{fig:ss_example} displays an example where using unsmoothed data performs poorly for these FIs, demonstrating why smoothing may be necessary for some scans.  This is a high-resolution scan, with voxel size 0.49mm$\times$0.49mm$\times$1, which may result in more noise in the image that may affect the performance of BET.  Moreover, in Table~\ref{tab:fail}, the estimated failure rates were lower using the smoothed data compared to the unsmoothed data.  We observe the lowest failure rate in the pipelines using smoothed data and an FI of $0.01$ or $0.1$.  Though this represents a large number of scans, failure rates may also be affected by patient-level characteristics, including the center where the patient was scanned.  

% latex table generated in R 3.1.1 by xtable 1.7-4 package
% Tue Dec 16 16:11:43 2014
\begin{table}[ht]
\centering
\begin{tabular}{cr|r}
  \hline & \multicolumn{2}{c}{Failure Scans: N (\%)} \\Fractional Intensity & Unsmoothed & Smoothed \\ 
  \hline
0.01 & 161 (14.7\%) & 57 (5.2\%) \\ 
  0.1 & 192 (17.5\%) & 80 (7.3\%) \\ 
  0.35 & 1068 (97.5\%) & 154 (14.1\%) \\ 
   \hline
\end{tabular}
\caption{{\bf Failure Rates for each Processing Pipeline of Brain Extraction of the 1095 Scans Analyzed.}} 
\label{tab:fail}
\end{table}


As multiple scanners were used, we wanted to determine if the failure rate was different across scanners.  In Table~\ref{tab:sfail}, we present the failure rate for each scanner, using smoothed data and an FI of $0.01$.  The failure rates for all scanner types other than Neurologica were below $6\%$.  Although we see a failure rate above $16\%$ in the Neurologica scanners, only $30$ scans were acquired using this type of scanner.  Moreover, a Fisher's exact test did not find a difference in the failure rates across scanner type ($p=0.110$).

% latex table generated in R 3.1.1 by xtable 1.7-4 package
% Tue Dec 16 16:11:43 2014
\begin{table}[ht]
\centering
\begin{tabular}{cc}
 Scanner Type & Failure Rate: Fail/N (\%) \\ 
  \hline
Siemens & 28/492 (5.7\%) \\ 
  GE & 15/298 (5.0\%) \\ 
  Philips & 7/207 (3.4\%) \\ 
  Toshiba & 2/66 (3.0\%) \\ 
  Neurologica & 5/30 (16.7\%) \\ 
  Picker & 0/2 (0.0\%) \\ 
   \hline
\end{tabular}
\caption{{\bf Failure Rates for Different Scanner Types using Smoothed Data and an FI of $0.01$ Processing Pipeline of Brain Extraction of the 1095 Scans Analyzed.}} 
\label{tab:sfail}
\end{table}


The ICC estimate was high using the successfully brain extracted scans from the smoothed data with an FI of $0.01$ (ICC: 0.929, 95\% CI: 0.91, 0.945) and $0.1$ (ICC: 0.928, 95\% CI: 0.909, 0.944).  In Figure~\ref{fig:icc}, we illustrate the ICV estimates, using an FI of $0.01$ and smoothed data, for successful brain extraction in scans 10 or fewer days post baseline scan (gray lines).  The black lines represent ICV estimates over time for $10$ randomly selected patients.  The blue line is a local regression (LOESS) \citep{cleveland_local_1992} line, which represents an estimate of the average ICV over time.  This LOESS line is relatively flat, indicating that the ICV estimate averaged over patients is stable.  We also observe that although within-patient variability exists for ICV estimates, the variability across patients is greater.  

\begin{figure}[htbp]
\centering
\includegraphics[width=.55\textwidth]{../results/{Intraclass_Correlation_no_crani_check_day10_black_0.01_truevol}.png}
\caption{{\bf Intracranial Volume (ICV) Estimates for Scans Less than 10 Days Post-Baseline.} These estimates are from the brain extraction pipeline using and FI of $0.01$ and smoothed data.  Each separate line represents an individual patient.  The black lines represent ICV estimates over time for $10$ randomly selected patients.  The blue line is a local regression (LOESS) scatterplot smoother.  
}
\label{fig:icc}
\end{figure}




\section{Conclusions}
Quantitative procedures based on data contained in head CT images require brain-only images for analysis. Indeed, intracranial volume estimation, intensity normalization, segmentation, and registration are a few examples of image processing steps that fundamentally rely on the existence of a reliable brain mask.  We have introduced the first validated automated brain extraction pipeline for head CT images. Validation was done using gold-standard manual segmentations of brain tissue and multiple measurements of intracranial volume per patient.  A novel finding is that smoothing the data using a conservative smoother ($1$mm$^3$ 3D Gaussian kernel) and using an FI of $0.01$ or $0.1$ provides good brain extraction for the sample studied.  These choices make a large difference in the performance of the algorithms and have not been previously reported in the literature.

Although the sample size was relatively small for the gold standard validation, the CT images used are from different people, different centers, and different scanners.  We have also shown that failure rates are low (5\%) using smoothed data and an FI of $0.01$ in a large number of scans.
Therefore, it is likely that the parameters and approaches described here will generalize well because CT scan data are expressed in standardized units (Hounsfield units) and multiple scanner types were investigated at multiple sites under multiple scanning parameters.  We are using a population of patients with intracranial hemorrhage and the accuracy of BET may be dependent on factors such as hematoma size, which may change the distribution of Hounsfield units or compress brain structures.  We observed good performance of BET in these patients using the parameters described, which may indicate even better performance for individuals with no observed pathology.

The research presented here is fully reproducible and we provide ready-to-use software for CT brain extraction. The R function designed to perform brain extraction is located at
\url{http://bit.ly/CTBET_RCODE} and example bash script for command-line FSL can be downloaded here \url{http://bit.ly/CTBET_BASH}.  As our software is publicly available and is based on open-source, free programs (FSL and R), our method is readily available to all users.

\section*{Acknowledgements}
We thank the patients and families who volunteered for this study and Genentech Inc. for the donation of the study drug (Alteplase).

\section*{Sources of Funding}
The project described was supported by the NIH grant RO1EB012547 from the National Institute of Biomedical Imaging And Bioengineering, T32AG000247 from the National Institute on Aging, R01NS046309, RO1NS060910, RO1NS085211, R01NS046309, U01NS080824 and U01NS062851 from the National Institute of Neurological Disorders and Stroke, and RO1MH095836 from the National Institute of Mental Health. Minimally Invasive Surgery and rt-PA in ICH Evacuation Phase II (MISTIE II) was supported by grants R01NS046309 and U01NS062851 awarded to Dr. Daniel Hanley from the National Institutes of Health (NIH)/National Institute of Neurological Disorders and Stroke (NINDS).  ICES was led by Co-Principal Investigator Dr. Paul Vespa at the University of California Los Angeles. Minimally Invasive Surgery and rt-PA in ICH Evacuation Phase III (MISTIE III) is supported by the grant U01 NS080824 awarded to Dr. Daniel Hanley from the National Institutes of Health (NIH)/National Institute of Neurological Disorders and Stroke (NINDS). Clot Lysis: Evaluating Accelerated Resolution of Intraventricular Hemorrhage Phase III (CLEAR III) is supported by the grant U01 NS062851 awarded to Dr. Daniel Hanley from the National Institutes of Health (NIH)/National Institute of Neurological Disorders and Stroke (NINDS). 

\newpage
\section*{Inline Supplementary Methods 1}
Let $I_{ia}, I_{im}$ be the indicators that voxel $i$ is labeled to be in the brain mask for the automatic and manual masks, respectively.  

A voxel $i$ is labeled to be a true positive (TP) when $I_{ia} = 1$ and $I_{im} = 1$, false positive (FP) when $I_{ia} = 1$ and $I_{im} = 0$, false negative (FN) when $I_{ia} = 0$ and $I_{im} = 1$, and true negative (TN) when $I_{ia} = 0$ and $I_{im} = 0$.  The number of true positive voxels is defined as: 
$$
\# \text{TP} = \sum_{i=1}^{V} \left( I_{ia} \times I_{im}\right)
$$
Sensitivity is defined as
$$
\frac{\# \text{TP} }{\# \text{TP} + \text{FN}} = \frac{ \sum_{i=1}^{V} \left( I_{ia} \times I_{im}\right) }{ \sum_{i=1}^{V} I_{im}},
$$
specificity is defined as
$$
\frac{\# \text{TN} }{\# \text{TN} + \text{FP}} = \frac{ \sum_{i=1}^{V} \left\{ (1-I_{ia}) \times (1- I_{im} ) \right\} }{ \sum_{i=1}^{V} (1 - I_{im} )},
$$
overall accuracy is defined as:
$$
\frac{\# \text{TN} + \text{TP} }{\# \text{TN} + \text{FN} + \text{TP} + \text{FP}} = \frac{ \sum_{i=1}^{V} \left[ (I_{ia} \times I_{im}) + \left\{ (1-I_{ia}) \times (1- I_{im} ) \right\} \right] }{\sum_{i=1}^{V} I_{ia}  + \sum_{i=1}^{V} I_{im}},
$$
and the Dice Similarity Index (DSI) is defined as
$$
\frac{2 \times \#\text{TP} }{ \# \text{TN} + \text{FN} + \text{TP} + \text{FP}} = \frac{ 2 \times \sum_{i=1}^{V} \left( I_{ia} \times I_{im}\right) }{\sum_{i=1}^{V} I_{ia}  + \sum_{i=1}^{V} I_{im} - \sum_{i=1}^{V} \left( I_{ia} \times I_{im}\right)}.
$$






\newpage
\section*{References}
\bibliographystyle{elsarticle-num-names}
\bibliography{CT_Skull_Stripping2}
%\printbibliography



\end{document}
